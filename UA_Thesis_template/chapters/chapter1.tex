\chapter{Introduction}
\label{chapter:introduction}

\section{Context and Motivation}

Nowadays, technological advances and continuous knowledge have established a permanent need for lifelong learning, such as specialization courses and, more recently, micro-credentials. This importance is based on the ability of such courses to keep professionals up to date with technological and knowledge developments, guaranteeing the updating of existing skills or the acquisition of new ones. However, the business sector is currently facing a challenge in trying to identify the specific training courses that can fill the gaps in the skills of its employees.

Currently, there are actions to co-create training offers between higher education institutions and companies belonging to the region's business community with the aim of creating training offers designed to keep up with the constant changes in the world of work.

However, the difficulty in linking a training offer (by its title, description or objectives) to the skills acquired at the end of its successful completion, still represents a barrier in the process of people and companies choosing a training offer.

To this end, the European Commission recently made available a database containing the multilingual taxonomy of European Skills, Competences, Qualifications and Occupations (ESCO)~\cite{what_esco}.
ESCO acts as a dictionary that describes, identifies and classifies professional occupations and skills relevant to the EU labour market and the education and training sectors, providing descriptions of 3008 occupations and 13 890 skills associated with these occupations, translated into 28 languages (those of the EU as well as Icelandic, Norwegian, Ukrainian and Arabic).

Even though great progress has been made in skills matching with the emergence of ESCO, currently, the API of the taxonomy is not enough, on its own, to always accurately return a list of ESCO skills for a given text of a training or educational offer, with some of the returned skills being not related at all with the input.

In order to address these gaps, several approaches will be discussed in this dissertation, especially using Large Language Models (LLMs), due to their powerful text processing capacities, role-playing ability and human language comprehension. Those will serve as inspiration for the development of the system'a pipeline.


\section{Objectives}
The main goal of this dissertation is to develop a computational system capable of processing training offers’ information, in this specific case, coming from UA micro-credentials and courses’ Pedagogical Dossiers (DPUCs) and map that data to ESCO skills, taking also advantage of Large Language Models (LLMs) and the ESCO API for that purpose. 

The objectives of this thesis work can be summarized as follows:
\begin{itemize}
   \item{Implementation and testing of a system to manage the skills of UA's educational offer and to match them to ESCO skills.}
   \item{Development of a pipeline that connects the ESCO framework (API), UA courses' Pedagogical Dossiers (DPUCs) and an LLM framework.}
   \item{Survey of the State of the Art regarding occupations and skills taxonomoies across the world, in the international, continental and national contexts.}
   \item{Study and further integration of ESCO API in the system's pipeline.}
   \item{Testing of several LLM frameworks and NLP libraries to test their applicability in the task of mapping UA's educational offer to ESCO skills.}
   \item{Deployment of the system's pipeline to automate the skills matching process.}
   \item{Evaluation of the system's performance, trusting on Course Directors' manual verification of the skills' matching.}
\end{itemize}

Therefore, the ultimate goal of this thesis is to provide the Aveiro's academic community with a trustable framework that maps the institution's educational offer to standardized and specific skills of an European occupations and skills taxonomy (ESCO). This will help current and future students to have a better understanding of the University's educational offer and companies' HR representatives to ackowledge and recognize UA's alumnis skills when employing them.

\section{Document Structure}

This dissertation is composed by two more chapters.
The next chapter, Chapter \ref{chapter:related work}, \nameref{chapter:related work} offers an overview of the existing taxonomies in the field of occupations and skills and a discussion about features of \acl{llms} that can be used to simplify the skills matching process. This chapter also provides the analysis of current approaches using these models for skill extraction.

In Chapter \ref{chapter:methodology}, \nameref{chapter:methodology}, the work developed throughout this pre-dissertation, as well as a projection of future work for the next semester is presented. This Chapter also includes a Gantt diagram containing the tasks that were and will be developed during this thesis work.